\documentclass{article}
\usepackage{graphicx}

\title{PROGRAMMING LANGUAGES}
\author{Love Folami}

\begin{document}
	\pagenumbering{gobble}
	\maketitle
	\newpage
	\section{PROGRAMMING LANGUAGES}
	\begin{itemize}
		\item JAVASCRIPT
		\item PYTHON
		\item RUBY
		\item PHP
		\item HTML (HYPERTEXT MARKUP LANGUAGE)
		\end{itemize}
		\subsection{Objectives}
		\begin{itemize}
			\item To understand the history of each of the programming language in this presentation.
            \item To learn about the founder(s) of each of the programming language in this presentation.
            \item Application examples that exist or can be developed from the programming languages.
            \item Available Integrated Development Environment (IDE’s) one can use when programming.
            \item Related programming languages.
		\end{itemize}
	
	\subsection{JAVASCRIPT}
	\begin{figure}[h!]
		\includegraphics[width=\linewidth]{js.png}
	\end{figure}
\subsubsection{HISTORY}
Brendan Eich, a Netscape programmer, created a new scripting language in ten days in September 1995. It was originally known as Mocha, but it was renamed Livescript, and eventually Javascript (JS).
Marc Andreeseen, the creator of Netscape Communications, intended to make the web more dynamic by making animations, user interaction and other forms of automation a normal element of each website, thus he created Javascript. Marc was also very much aware that Microsoft was working on their own browser, Internet Explorer, so he sought to make Netscape Communicator, his browser, more appealing to developers by including both an enterprise-level coding language (Java) and a smaller scripting language (Javascript).

JavaScript did not start off well and those who developed Java thought of it as more of a “UI glue” (which is a programming language with a user interface (UI) design that is designed to write and manage programs and codes, and also connects different software components) to be used often by designers and other non-engineers. However, having a “glue language allowed the internet to grow, as programmers were able to respond more swiftly to events and create interactive components. As a result, Javascript spread rapidly and swiftly became the internet’s common language. 
 \subsubsection{FOUNDER}
 The founder of JavaScript as seen earlier in the presentation is Brandan Eich.
\subsubsection{APPLICATION EXAMPLES THAT EXISTS OR CAN BE DEVELOPED.}
The following are instances of Javascript usage and applications
\begin{itemize}
	\item Web apps
\item Presentation
\item Server apps
\item Mobile apps
\item Games Designing
\end{itemize}
\paragraph{PRESENTATION:}
Developers can use RevealJS and BespokeJS, two Javascript frameworks, to create a slide deck (which is a collection of slides in a presentation) on the web. RevealJS is an HTML presentation framework that includes touch support in its code. As a result, mobile devices such as phones and tablets may access online presentation, while the BespokeJS plugin is a functionality framework that provides properties such as scaling and animated bullets when scripting and BespokeJS is described as being lightweight.
\paragraph{SERVER APPS:}
Node.js is the the most often used runtime environment for JS, and it allows developers to write, test, and debug codes as well as write server-side software. For example, Opera Unite which is a feature of the Opera Browser. And users can use the browsers to execute server applications such as file sharing and streaming.
\paragraph{WEB APPS:}
Angular and Vue.js are two popular JS frameworks used b developersfor creating apps. AngularJS and API’s were used to create Netflix and Paypal. API (Application Programming Interface) is a web-based software access protocol.
\paragraph{GAMES DESIGNING:}
EaselJS, a library recognized for its excellent graphics, is commonly found in JS games. For web game development, JS and HTML5 are a wonderful match. HTML5 gives you complete access to the web without the need for additional plugins such as Flash. As a result, it is cross-platformed, meaning you won’t have to switch devices to see a full web page.
\paragraph{MOBILE APPS:}
These are designed to be local apps with no connection to the internet . For this style of development, React Native and ReactJS are employed. 
React Native is useful for creating mobile apps since it allows for the intgration of native functionality into blended apps. Blended apps are web-based, yet they may be launched directly from a mobile app system without the need to open a browser. ReactJS is in charge of creating user interfaces.
\subsubsection{AVAILABLE IDE’s (INTEGRATED DEVELOPMENT ENVIRONMENT) FOR JS}
An IDE is a coding environment that helps with web and app devlopment. Some available IDE’s for JS include:
\begin{itemize}
	\item WebStorm
\item Komodo IDE
\item Atom
\item Sublime Text
\item Visual Studio Code
\item Brackets
\item Apache Netbeans
\item CodeAnywhere
\item Eclipse
\end{itemize}
\subsubsection{PROGRAMMING LANGUAGES SIMILAR TO JS}
\begin{itemize}
	\item Python
\item Ruby
\end{itemize}

\subsection{PYTHON}
\begin{figure}[h!]
	\includegraphics[width=\linewidth]{python.jpg}
	\end{figure}
\subsubsection{HISTORY}
Python designed by Guido van Rossum is a widely used, interpreted and high level programming language for general-purpose programming, which was initially released on February 20th,1991. Python takes its name from Monty Python’s Flying Circus, an old BBC sketch comedy show.
In1999, Guido van Rossum vision for Python was:
\begin{itemize}
	\item a simple and clear language that is just as powerful as the major competitors.
\item it would be an open source so that anyone can contribute to its growth.
\item code that is simple to grasp just like plain English.
\item ideal for day-to-day tasks, with the ability to perform small development tasks.
\end{itemize}
Presently, it is obvious that these goals have ben achieved. The latest version of Python is Python 3. 
\subsubsection{FOUNDER}
The founder of this programming language is Guido Van Rossum.
\subsubsection{APPLICATION EXAMPLES THAT EXISTS OR CAN BE DEVELOPED}
\begin{itemize}
	\item Web Creation
\item Games Designing
\item Desktop GUI
\item Business Application
\item Audio and Video Application
\end{itemize}
 \paragraph{WEB CREATION:}
 Python includes a number of libraries that can aid in the integration of protocols such as HTTPS, FTP, and others during the web building process. Python can also be used to process XML (extensible markup language), E-mail, and a variety of other formats.
\paragraph{GAMES DESIGNING:}
Python is a programming language that is used to create interactive games. PySoy, a 3D gaming engine that supports Python 3, is one of these libraries. Python has been used to create games such as Vega Strike, Civilization IV, and others.
\paragraph{DESKTOP GRAPHIC UI:}
Python is the programming language that we use to create desktop applications. It contains the Tkinter library, which is used to create user interfaces. Other relevant toolkits include Kivy and PYQT, which can be used to construct apps for a variety of platforms.
\paragraph{BUSINESS APPLICATIONS:}
Accessible, expandable, and legible platforms are required for business applications, and Python delivers platforms that meet all of these requirements. Tryton is an example of a Python-based platform.
\paragraph{AUDIO AND VIDEO APPLICATIONS:}
Python is used to create multi-tasking programs that can also produce media.
Python libraries are used to create audio and video programs like TimPlayer and Cplay, which give superior stability and performance than other media players.
\subsubsection{AVAILABLE IDE’s FOR PYTHON}
\begin{itemize}
	\item Pydev
\item Pycharm
\item Visual Studio Code
\item Atom
\item IDLE
\item Spyder
\end{itemize}
\subsubsection{PROGRAMMING LANGUAGES SIMILAR TO PYTHON}
\begin{itemize}
	\item Ruby
\end{itemize}

\subsection{RUBY}
\begin{figure}[h!]
	\includegraphics[width=\linewidth]{ruby.jpg}
	\end{figure}
\subsubsection{HISTORY}
Yukihiro Matsumoto designed Ruby in the mid-1990s. He is known as "Matz" in the Ruby community for being the father of the Ruby programming language. Ruby was first launched in 1995, with Ruby.95 being the first version to be released. Yukihiro sought an object-oriented programming language that could also be used as a scripting language, therefore he created Ruby.
\subsubsection{FOUNDER}
The founder of this programming language is Yukihiro Matsumoto.
\subsubsection{APPLICATION EXAMPLES THAT EXISTS OR CAN BE DEVELOPED}
\begin{itemize}
	\item Web Apps
\item E-Commerce
\item Data Control
\item Custom Database solutions
\item Prototyping
\end{itemize}
\paragraph{WEB APPS:}
Ruby is a basic and easy-to-understand programming language that can be used to create a wide range of web apps.
\paragraph{DATA CONTROL:}
Ruby is used to create content-centric websites since it has a large number of built-in libraries that allow you to quickly create a site.
\paragraph{CUSTOM DATABASE SOLUTIONS:}
Ruby on Rails, a framework based on the Ruby programming language, employs ActiveRecord to enable for easy database management without the usage of SQL. It also connects with a database management system, giving you the ability to manage large databases.
\paragraph{PROTOTYPING:}
Before working on a complete application, Ruby can be used to develop test versions. It also allows you to work on a rough drawing that you can display to the rest of your team so they can understand how the final software should look.
\subsubsection{AVAILABLE IDE’s FOR RUBY}
\begin{itemize}
	\item RubyMine
\item Aptana Studio
\item Cloud9
\item NetBeans
\end{itemize}
\subsubsection{PROGRAMMING LANGUAGES SIMILAR TO RUBY}
\begin{itemize}
	\item Python
\end{itemize}

\subsection{PHP}
\begin{figure}[h!]
	\includegraphics[width=\linewidth]{php.jpg} 
	\end{figure}
\subsubsection{HISTORY}
PHP stands for Hypertext Preprocessor, and it was previously known as Personal Home Page. It's a general-purpose programming language for building websites and web applications. It's an HTML-enabled server-side scripting language. Rasmus Lerdorf designed it in 1994.
PHP started out as a simple set of Common Gateway Interface (CGI) binaries written in the C computer language. Rasmus entitled the scripts Personal Home Page Tools since they were first used to track visits to his online CV. More functionality was required over time, so he redesigned the PHP Tools, resulting in a much larger and more robust version. This new model could connect with databases and provided a foundation for users to create simple dynamic web apps.
\subsubsection{FOUNDER}
Rasmus Lerdorf is the programming language inventor.
\subsubsection{APPLICATION EXAMPLES THAT EXISTS OR CAN BE DEVELOPED}
\begin{itemize}
	\item Websites or Web Applications
\item Desktop (GUI) Applications
\item Command line scripting
\item Gaming web applications
\end{itemize}
\subsubsection{AVAILABLE IDE’s FOR PHP}
\begin{itemize}
	\item Eclipse
\item NetBeans
\item PhpStorm
\item Angular
\item Komodo IDE
\item CodeAnywhere
\end{itemize}
\subsubsection{PROGRAMMING LANGUAGES SIMILAR TO PHP}
\begin{itemize}
	\item Java
\item C programming language
\item HTML
\end{itemize}

\subsection{HTML}
\begin{figure}[h!]
	\includegraphics[width=\linewidth]{html.png}
	\end{figure}
\subsubsection{HISTORY}
Sir Tim Berners-Lee developed HTML in late 1991, but it was never officially published. HTML 2.0 was first released in 1995. HTML 4.01 was a major version of HTML released in late 1999. Hypertext Markup Language is the comprehensive definition of HTML.
HTML 5.0, which was launched in 2012, is currently in use all around the world.
\subsubsection{FOUNDER}
Sir Tim Berners-Lee is the originator of HTML.
\subsubsection{APPLICATION EXAMPLES THAT EXISTS OR CAN BE DEVELOPED}
\begin{itemize}
	\item Webpage Development
\item Web Document creation
\item Internet Navigation
\item Internet Navigation
\item Offline capabilities usage
\item Game Development usage
\end{itemize}
\subsubsection{AVAILABLE IDE’s FOR HTML}
\begin{itemize}
	\item Sublime Text
\item Apana Studio 3
\item PyCharm
\item RubyMine
\item Adobe Dreamweaver CC
\end{itemize}
\subsubsection{PROGRAMMING LANGUAGES SIMILAR TO HTML}
\begin{itemize}
	\item PHP
\end{itemize}
\thispagestyle{empty}
\pagenumbering{arabic}
\end{document}